\documentclass[
a4paper, 
10pt         %% set default font size to 10 point
]{article}  %% article, see KOMA documentation (scrguide.dvi)

%%%%%%%%%%%%%%%%%%%%%%%%%%%%%%%%%%%%%%%%%%%%%%%%%%%%%%%%%%%%%%%%%%%%%%%%%%%%%%%%
%%%
%%% packages
%%%

%%%
%%% encoding and language set
%%%

%%% inputenc: coding of german special characters
\usepackage[latin1]{inputenc}

%%% fontenc, ae, aecompl: coding of characters in PDF documents
\usepackage[T1]{fontenc}
\usepackage{ae,aecompl}

%%%
%%% technical packages
%%%

%%% amsmath, amssymb, amstext: support for mathematics
%\usepackage{amsmath,amssymb,amstext}

%%% psfrag: replace PostScript fonts
\usepackage{psfrag}
\usepackage{geometry}
 \geometry{
 a4paper,
 left=20mm,
 top=20mm,
 }
%%% listings: include programming code
%\usepackage{listings}

%%% units: technical units
%\usepackage{units}

%%%
%%% layout
%%%

%%%
%%% PDF
%%%

\usepackage{ifpdf}

%%% Should be LAST usepackage-call!
%%% For docu on that, see reference on package ``hyperref''
\ifpdf%   (definitions for using pdflatex instead of latex)

  %%% graphicx: support for graphics
  \usepackage[pdftex]{graphicx}
  \usepackage{wrapfig}

  \pdfcompresslevel=9

  %%% hyperref (hyperlinks in PDF): for more options or more detailed
  %%%          explanations, see the documentation of the hyperref-package
  \usepackage[%
    %%% general options
    pdftex=true,      %% sets up hyperref for use with the pdftex program
    %plainpages=false, %% set it to false, if pdflatex complains: ``destination with same identifier already exists''
    %
    %%% extension options
    backref,      %% adds a backlink text to the end of each item in the bibliography
    pagebackref=false, %% if true, creates backward references as a list of page numbers in the bibliography
    colorlinks=true,   %% turn on colored links (true is better for on-screen reading, false is better for printout versions)
    %
    %%% PDF-specific display options
    bookmarks=true,          %% if true, generate PDF bookmarks (requires two passes of pdflatex)
    bookmarksopen=false,     %% if true, show all PDF bookmarks expanded
    bookmarksnumbered=false, %% if true, add the section numbers to the bookmarks
    %pdfstartpage={1},        %% determines, on which page the PDF file is opened
    pdfpagemode=None         %% None, UseOutlines (=show bookmarks), UseThumbs (show thumbnails), FullScreen
  ]{hyperref}


  %%% provide all graphics (also) in this format, so you don't have
  %%% to add the file extensions to the \includegraphics-command
  %%% and/or you don't have to distinguish between generating
  %%% dvi/ps (through latex) and pdf (through pdflatex)
  \DeclareGraphicsExtensions{.pdf}

\else %else   (definitions for using latex instead of pdflatex)

  \usepackage[dvips]{graphicx}

  \DeclareGraphicsExtensions{.eps}

  \usepackage[%
    dvips,           %% sets up hyperref for use with the dvips driver
    colorlinks=false %% better for printout version; almost every hyperref-extension is eliminated by using dvips
  ]{hyperref}

\fi


%%% sets the PDF-Information options
%%% (see fields in Acrobat Reader: ``File -> Document properties -> Summary'')
%%% Note: this method is better than as options of the hyperref-package (options are expanded correctly)
\hypersetup{
  pdftitle={}, %%
  pdfauthor={}, %%
  pdfsubject={}, %%
  pdfcreator={Accomplished with LaTeX2e and pdfLaTeX with hyperref-package.}, %% 
  pdfproducer={}, %%
  pdfkeywords={} %%
}


%%%%%%%%%%%%%%%%%%%%%%%%%%%%%%%%%%%%%%%%%%%%%%%%%%%%%%%%%%%%%%%%%%%%%%%%%%%%%%%%
%%%
%%% user defined commands
%%%

%%% \mygraphics{}{}{}
%% usage:   \mygraphics{width}{filename_without_extension}{caption}
%% example: \mygraphics{0.7\textwidth}{rolling_grandma}{This is my grandmother on inlinescates}
%% requires: package graphicx
%% provides: including centered pictures/graphics with a boldfaced caption below
%% 
\newcommand{\mygraphics}[3]{
  \begin{center}
    \includegraphics[width=#1, keepaspectratio=true]{#2} \\
    \textbf{#3}
  \end{center}
}

%%%%%%%%%%%%%%%%%%%%%%%%%%%%%%%%%%%%%%%%%%%%%%%%%%%%%%%%%%%%%%%%%%%%%%%%%%%%%%%%
%%%
%%% define the titlepage
%%%

% \subject{}   %% subject which appears above titlehead
% \titlehead{} %% special heading for the titlepage

%%% title
\title{Damage detection on stones}

%%% author(s)
\author{Samirah Amadu \and
Javier Lara Juarez }

%%% date
\date{}

%%%%%%%%%%%%%%%%%%%%%%%%%%%%%%%%%%%%%%%%%%%%%%%%%%%%%%%%%%%%%%%%%%%%%%%%%%%%%%%%
%%%
%%% begin document
%%%

\begin{document}

 \maketitle


%%%%%%%%%%%%%%%%%%%%%%%%%%%%%%%%%%%%%%%%%%%%%%%%%%%%%%%%%%%%%%%%%%%%%%%%%%%%%%%%
%%%
%%% begin main document
%%% structure: \section \subsection \subsubsection \paragraph \subparagraph
%%%

\section{Problem statement}
In a common stone refinement pipeline, quality assurance is done by humans, causing a non-uniform selection in damaged stones. Furthermore, the most expensive component within a pipeline is human work. To save money and to optimize the pipeline a stone production company would like to automate the process of damage detection and the sorting of identified stones.\\

This project deals with the first step, the automated detection and localization of damage on stones.

\section{Dataset}
\begin{wrapfigure}[13]{r}{0.5\textwidth}
 \vspace{-20pt}
  \begin{center}
    \includegraphics[width=0.5\textwidth]{example.jpg}
  \end{center}
  \vspace{-15pt}
  \caption{A picture of a damaged stone.}
\end{wrapfigure}

The dataset consists of 187 black and white images of different square shaped stones, with different levels of damage. As part of the preprocessing each of the images will be cropped and labelled as undamaged or damaged.\\
As of today it is uncertain if the amount of data available is sufficient as such it may be necessary to use data augmentation to artificially increase the size of the dataset.

\section{Solution}
The main focus of this project is to train a neural network, such that damaged stones can be identified. \\

The CNN ResNet will be used to generate a model that can achieve classification. Once this step has been successful the localization part of the project is initiated using a region based CNN. The performance of the classification will be assessed using cross entropy and the localization performance using $L_{2}$ distance.\\

In terms of implementation set-up we well be using Keras and Tensorflow on an Anaconda environment. 


\section{Expected results}
The result of this project will be a trained neural network model which is capable of identifying damaged stones and localizing the damaged area with high accuracy and efficiency.  







\end{document}
